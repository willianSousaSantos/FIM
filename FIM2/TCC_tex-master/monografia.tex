
\documentclass[monografia]{tcc-abntex2}
%http://www.eac.ufsm.br/pesquisa/qualidade-sonora


% ---
% PACOTES
% ---
%\usepackage{cmap}				% Mapear caracteres especiais no PDF
\usepackage{pslatex}			% Usa a fonte Times News Roman			
\usepackage{makeidx}            % Cria o indice
\usepackage{hyperref}  			% Controla a formação do índice
\usepackage{lastpage}			% Usado pela Ficha catalográfica
\usepackage{indentfirst}		% Indenta o primeiro parágrafo de cada seção.
\usepackage{nomencl} 			% Lista de simbolos
\usepackage{graphicx}			% Inclusão de gráficos
\usepackage{lipsum}				% para geração de dummy text
\usepackage{float}
\usepackage[printonlyused]{acronym}
\usepackage{algpseudocode,algorithm}
\usepackage{multicol}
\usepackage{amsfonts}
\usepackage{mathrsfs}
%\usepackage{vaucanson-g}
%\usepackage[table]{xcolor}

%criar um novo estilo de cabeçalhos e rodapés
\makepagestyle{tccraiza}
  %cabeçalhos
  \makeevenhead{tccraiza} %pagina par
     {}
     {}
     {\thepage}
  \makeoddhead{tccraiza} %pagina ímpar ou com oneside
     {}
     {}
     {\thepage}
  \makeheadrule{tccraiza}{\textwidth}{\normalrulethickness}
  % rodapé
  %\makeevenfoot{tccraiza}
     %{rodapé par à esquerda} %pagina par
     %{centro \thepage}
     %{direita} 
  %\makeoddfoot{tccraiza} %pagina ímpar ou com oneside
    % {rodapé ímpar/onside à esquerda}
     %{centro \thepage}
     %{direita}


% ----------------------------------------------------------
\makeindex
% ----------------------------------------------------------
% Compila a lista de abreviaturas e siglas
% ----------------------------------------------------------
\makenomenclature
% ----------------------------------------------------------
% Inserir folha de aprovação digitalizada com as assinaturas
% ----------------------------------------------------------
%\inserirfolhaaprovacao{folhaAprovacao.pdf}


% ----------------------------------------------------------
% Início do documento
% ----------------------------------------------------------

\begin{document}



% ----------------------------------------------------------
% ELEMENTOS PRÉ-TEXTUAIS
% ----------------------------------------------------------
\pretextual
\pagenumbering{roman}
% Insere Capa, Folha de rosto e folha de aprovação (se inserida).
\thispagestyle{empty}
\begin{center}

------------------------------------------------------------------------------------------------------------
\normalsize{Curso de Ciência da Computação \\ Universidade Estadual de Mato Grosso do Sul}
------------------------------------------------------------------------------------------------------------

\vspace*{3cm}

\Large{ESTUDO E ANÁLISE DE MÉTODOS PARA RECONHECIMENTO DE PALAVRAS DITAS}


\vspace*{3cm}
\normalsize{Raiza Artemam de Oliveira \\ Willian de Sousa Santos}

\vspace*{2cm}

\normalsize{Prof. MSc. André Chastel Lima (Orientador)}

\vspace*{6cm}


\large{DOURADOS-MS \\ 2016}
\end{center}
\include{preTextual/contracapa}
\thispagestyle{empty}
\begin{center}

\Large{Estudo e Análise de Métodos para Reconhecimento de Palavras Ditas}

\vspace*{4cm}

\large{Raiza Artemam de Oliveira \\ Willian de Sousa Santos }

\vspace*{4cm}

\end{center}
\begin{flushright}
\begin{minipage}{0.5\textwidth}
\normalsize{
Este exemplar corresponde à redação final
da monografia da disciplina Projeto Final de Curso 
devidamente corrigida e defendida por
 Raiza Artemam de Oliveira e Willian de Sousa Santos
e aprovada pela Banca Examinadora, 
como parte dos requisitos para a obtenção
do título de Bacharel em Ciência da Computação.

\vspace*{2cm}

Dourados, xx de novembro de 2016

\vspace*{2cm}

Prof. MSc. André Chastel de Lima
}
\end{minipage}
\end{flushright}




\thispagestyle{plain}
\begin{center}

------------------------------------------------------------------------------------------------------------
\normalsize{Curso de Ciência da Computação \\ Universidade Estadual de Mato Grosso do Sul}
------------------------------------------------------------------------------------------------------------

\vspace*{2cm}

\LARGE{ESTUDO E ANÁLISE DE MÉTODOS PARA RECONHECIMENTO DE PALAVRAS DITAS}


\vspace*{1.5cm}
\normalsize{Raiza Artemam de Oliveira \\ Willian Sousa Santos}

\vspace*{1.5cm}

\normalsize{Outubro de 2016}

\vspace*{3cm}
\end{center}
\begin{flushleft}
\begin{minipage}{0.5\textwidth}
\normalsize{
BANCA EXAMINADORA:\\


Prof. MSc. André Chastel Lima (Orientador) \\
Área de Computação – UEMS\\



Profa. [Titulação]  Nome do professor \\
 Área de Computação – UEMS\\


 Profa. [Titulação] Nome do professor \\
 Área de Computação – UEMS 
}
\end{minipage}
\end{flushleft}



% Pequena dedicatpertinente ao seu 
% trabalho ou que represente o seu modo de pensar.) 
% 
%
% Arquivo: dedicatoria.tex
% ----------------------------------------------------------------------- %
\thispagestyle{plain}
\vspace*{\fill}

{ \raggedleft


\textit{Computer science is no more about computers than astronomy is about telescopes, biology is about microscopes or chemistry is about beakers and test tubes. Science is not about tools, it is about how we use them and what we find  out when we do.. \\
Edgar Dijkstra}

~
}
\normalsize{Gostariamos de agradecer ao professor André Chastel Lima pela dedicação e paciência  durante o desenvolvimento deste trabalho. A professora Maria de Fátima pela ajuda no desenvolvimento do texto. Ao professor coordenador Nilton Cezar de Paula. A secretária dona Jandira pela atenção dedicada as nossas vidas acadêmicas. Por fim agradecemos a todos os professores que contribuiram nessa jornada.\\

\quad Eu, Raiza, agradeço aos meus pais, Eneias e Sandra, por todo o apoio, generosidade, educação e valores que me ensinaram. Aos meus tios, Marcia e Juraci, por serem sempre prestativos e generosos comigo. Aos meus avós, Georgina e Otávio, e , Maria Lucilene (\textit{in memorian}) e João. Por fim agradeço a professora Adriana Betania de Paula Molgora por sua orientação nos meus primeiros passos na vida acadêmica e por propiciar a oportunidade de desenvolver projetos de iniciação cientifica.  


%%%%%%%%%%%%%
Gostariamos de agradecer ao professor André Chastel Lima pela dedicação e paciência  durante o desenvolvimento deste trabalho. A professora Maria de Fátima pela ajuda no desenvolvimento do texto. Ao professor coordenador Nilton Cezar de Paula. A secretária dona Jandira pela atenção dedicada as nossas vidas acadêmicas. Por fim agradecemos a todos os professores que contribuiram nessa jornada.\\

\quad Eu, Raiza, agradeço aos meus pais, Eneias e Sandra, por todo o apoio, generosidade, educação e valores que me ensinaram. Aos meus tios, Marcia e Juraci, por serem sempre prestativos e generosos comigo. Aos meus avós, Georgina e Otávio, e , Maria Lucilene (\textit{in memorian}) e João. Por fim agradeço a professora Adriana Betania de Paula Molgora por sua orientação nos meus primeiros passos na vida acadêmica e por propiciar a oportunidade de desenvolver projetos de iniciação cientifica.  


}
% ---
%\maketitle

\begin{resumo}{MFCC. SOM. HMM. ALSA. Reconhecimento de Fala}
\label{sec:resumo}
O  reconhecimento de fala é um campo de estudo muito amplo e com diversas etapas envolvidas. O processo de reconhecimento de uma palavra isolada se inicia na captação da onda sonora, passa por vários métodos para tratamento desta onda e extração de descritores do sinal. Uma vez obtido estes descritores se inicia a fase de comparação. Nesta fase, um algoritmo deve ser aplicado para o reconhecimento do sinal de entrada. Existem vários algoritmos que utilizam técnicas diferentes. Para chegar a uma solução no reconhecimento, além dos algoritmos e descritores, o ambiente, o contexto da aplicação e os recursos disponíveis também devem ser considerados na decisão de quais técnicas devem ser empregadas. Neste trabalho trazemos uma introdução a estas técnicas. São estudados métodos desde a primeira etapa que consiste na captura e processamento do sinal digital até a fase final, onde são considerados os algoritmos e técnicas aplicados ao reconhecimento de uma palavra isolada.
\end{resumo}
\begin{abstract}{MFCC. HMM. Alsa. Speech Recognition}
\label{sec:abstract}
Speech recognition is a very broad field of study and with several steps involved. The recognition process of an isolated word starts in the capture of the sound wave passes through several methods for the treatment of this wave signal and extracting features. Once obtained these features starts the comparison phase. At this stage an algorithm must be applied to the recognition of the input signal. There are several algorithms that use different techniques. To get a solution in addition to the recognition algorithms and features extraction, the environment, the application context and the resources available should also be considered in deciding which techniques should be employed. In this paper we bring an introduction to these techniques. The methods are studied from the first step of the capture and digital signal processing to the final stage, which are considered the algorithms and techniques applied to the recognition of a isolated word.
\end{abstract}
\sumario
\listasiglas{abrev/Abreviaturas}
\listatabelas
\listafiguras


% ----------------------------------------------------------
% ELEMENTOS TEXTUAIS
% ----------------------------------------------------------
\pagenumbering{arabic} % volta à numeração arábica
\pagestyle{tccraiza}
% ----------------------------------------------------------
% Introdução, Desenvolvimento e Conclusão
% ----------------------------------------------------------
\chapter{INTRODUÇÃO}
\label{chap:introducao}
 \thispagestyle{plain}


\quad Nos primeiros sistemas computacionais a comunicação entre pessoas e máquinas era realizada através de terminais por linha de comando.  Apenas especialistas conseguiam utilizar estes sistemas.
Depois, no início da década de 70, com a criação do mouse e a introdução da interface gráfica  os sistemas tornaram-se mais amigáveis ao usuário, podendo ser utilizados por pessoas comuns sem necessidade de conhecimento técnico. Com o passar dos anos a interação entre pessoas e máquinas tornou-se mais intuitiva com às diversas interfaces entre o usuário e o sistema. No fim da década de 70 iniciaram-se as pesquisas de reconhecimento de fala.
Interfaces por meio de fala são utilizadas em diversas áreas, tais como: sistemas embarcados, automação residencial, operações bancárias, conversão fala texto e dispositivos móveis.

\quad O reconhecimento da fala é um campo de estudo amplo e necessário as diversas tecnologias que utilizam
desta como um  meio de comunicação entre o usuário e o sistema. Utilizar a fala como entrada de um sistema
torna a comunicação entre o usuário e o sitema mais direta, intuitiva, rápida e precisa. Como um campo de ampla aplicacação, o reconhecimento de fala tem diversos projetos em diferentes partes do mundo. Dentre os quais se destaca o projeto CMU Sphinx da universidade americana Carnegie Mellon. O projeto já tem cerca de 20 anos de pesquisas na área de reconhecimento de fala e de voz. Trata-se de um projeto open source voltado para linux, mas também conta com uma versão em java multiplataforma. O CMU Sphinx oferece suporte para várias linguagens, dentre elas o inglês, alemão, russo, francês e espanhol. 
O reconhecimento de fala pode ser classificado de acordo com o tamanho do vocábulario, de acordo com os algoritmos utilizados e de acordo com o tipo de fala a ser reconhecida (contínua ou discreta).



%A introdução deve conter a delimitação do tema, o problema, a justificativa e o
%objetivo do projeto, que podem vir em subseções separadas ou não.
%É muito importante ressaltar que a delimitação do tema requer clareza a respeito do
%campo de conhecimento a que pertence o assunto. O problema é o objeto de pesquisa ou de
%estudo. Optou-se, neste exemplo, em separar em subseções a justificativa e o(s) objetivo(s).\\
%No caso de projeto de pesquisa, que esteja vinculado a um grupo de pesquisa
%institucional, neste item é necessário acrescentar a denominação do grupo, que esteja
%devidamente certificado pela Unifra, e a denominação da linha de pesquisa a que pertence o
%projeto

\section{Justificativa}
\label{sec:justificativa}



O reconhecimento de palavaras ditas é um campo de estudo de extrema importância para uma melhor comunicação entre usuários e sistema.
%Na justificativa mencionam-se a relevância científica do trabalho, a contribuição da
%pesquisa e que benefício poderá trazer à comunidade ou à sociedade. Ainda devem estar claros
%o motivo da escolha do tema e as possibilidades de realização da pesquisa.
\section{Objetivos}
\label{sec:objetivos}

O objetivo deste trabalho é estudar os principais métodos de reconhecimento de fala. Analisar os algoritmos utilizados, suas vantagens e desvantagens. Apresentar os resultados para um pequeno vocábulario.
%A definição dos objetivos determina o que se quer atingir com a realização do
%trabalho de pesquisa. Objetivo é sinônimo de meta, fim.
%Uma sugestão interessante, na redação dos objetivos, é utilizar, no início das
%sentenças, o verbo no infinitivo, tais como: esclarecer tal coisa, definir tal assunto, procurar
%aquilo, permitir algo, demonstrar alguma coisa, entre outros.
%Alguns autores separam os objetivos em objetivo geral e objetivos específicos, mas
%não há regra a ser cumprida quanto a isso. Caso se opte em separá-los, tem-se:
\subsection{Objetivo geral}
\label{subsec:objetivogeral}
Estudar e analisar os algoritmos existentes para o reconhecimento de palavras  ditas em um vocábulario pequeno e um ambiente não controlado.
%O objetivo geral vincula-se à própria significação geral do tema proposto pelo
%projeto, ou seja, significa traçar as principais metas que norteiam a pesquisa.
\subsection{Objetivo específico}
\label{subsec:objetivoespecifico}
%Descrever aqui o(s) propósito(s) específico(s) para atingir um ponto de vista do tema,
%um ângulo a ser pesquisado, permitindo atingir o objetivo geral. Aconselha-se, na redação
%desta seção, não ser prolixo.
Apontar a melhor solução para reconhecimento de palavras ditas em ambientes não controlados.


\section{Metodologia}

\quad A metodologia adotada para a realização deste trabalho consiste nos seguintes passos:

\begin{itemize}
\item Pesquisa em livros, sites, artigos e notas de aula sobre o tema abordado e seus diversos aspectos;
\item Estudo de algoritmos aplicados ao reconhecimento de fala;
\item Implementação computacional de algoritmos aplicados ao reconhecimento de fala;
\item Testes e validação dos algoritmos implementados;
\item Análise e validação dos resultados obtidos com os métodos implementados;
\item Documentação do trabalho. 

\end{itemize}

\quad No capítulo \ref{chap:referencial_teorico} é feita uma explicação do que é relevante para este trabalho com base na literatura. No capítulo \ref{chap:captura_de_áudio} é feita uma explicação de como é realizada a captação de áudio, até a transformação em coeficientes mel-cepstrais. No capítulo \ref{chap:Modelos_Ocultos_de_Markov} contém a definição de modelos ocultos de markov, junto com uma breve descrição dos 3 problemas de Markov. No capítulo \ref{chap:Redes_neurais} Possui uma breve introdução sobre o que é redes neurais, junto com a definição de mapas auto-organizáveis. No capítulo \ref{chap:Implementaçaos} é descrito as ferramentas,linguagens de programação e scritps que foram utilizados para a implementação. No capítulo \ref{Testes_e_Análises} tabelas referentes aos dados de entrada,  processamento e os resultados obtidos.


























\chapter{FUNDAMENTAÇÃO TEÓRICA}
\label{chap:referencial_teorico}
\thispagestyle{plain}
Este capítulo traz uma revisão bibliográfica dos sistemas de reconhecimento de fala e processamento de sinais digitais, uma vez que estes são os assuntos que constituem a base deste trabalho.

De acordo com \cite{fundRecFala}, os  sistemas de reconhecimento de fala podem ser classificados em três grupos de acordo com a técnica utilizada. Estes grupos são :
\begin{itemize}
\item Reconhecedores por inteligência artificial;
\item Reconhecedores por comparação de padrões;
\item Reconhecedores baseados na análise acústico-fonética.
\end{itemize}
\section{Sistemas de reconhecimento de fala }

\label{sec:sistemasdereconhecimentodefala}


\subsection{Reconhecedores baseados em inteligência artificial}

Os sistemas de reconhecimento de fala que utilizam a inteligência artificial usam propriedades tanto dos reconhecedores por comparação de padrões quanto dos reconhecedores baseados na análise acústico-fonética. Sistemas com redes neurais são encaixados nesta classe. As redes Multilayer Perceptron usam uma matriz de ponderação que representa as conexões entre os nós da rede, e cada saída está associada a uma unidade a ser reconhecida \cite{kluwerNeural}.

A abordagem de inteligência artificial  baseia-se no processo humano natural de ouvir, analisar e tomar uma decisão sobre as características acústicas medidas para reconher a fala. Faz parte do processo de reconheciemento de fala pela abordagem de inteligência artificial o processo de segmentação e rotulagem usado na análise acústico-fonética  \cite{fundRecFala}. Esta abordagem aplica o conceito de que o conhecimento é dinâmico  e os modelos devem adaptar-se frequentemente. 

\subsection{Reconhecedores por comparação de padrões}

Estes reconhecedores usam o príncipio de que o sistema foi treinado para reconhecer os padrões. Os sistemas por reconhecimento de padrões possuem duas fases diferentes :
\begin{itemize}
\item Treinamento;
\item Reconhecimento.
\end{itemize}

Durante a fase de treinamento são criados padrões de referência para o sitema. Na fase de reconhecimento compara-se os padrões obtidos com os padrões de referência criados na fase anterior e calcula-se uma medida de similaridade entre os padrões. O padrão mais similar ao desconheido é escolhido como reconhecido. Os sistemas que se baseiam nos Modelos Ocultos de Markov (HMM) se encaixam nesta categoria.\\

Dentre as diversas razões para usar a abordagem de comparação de padrões para reconheimento de fala pode-se citar a simplicidade de uso, por ser um método de fácil entendimento que possui uma rica fundamentação matemática e é amplamente utilizado.  Destaca-se também a robustez, como um método robusto e invariante para diferentes vocábularios, algoritmos de comparação de padrão e regras de decisão. Isto torna esta abordagem apropriada para uma vasta gama de unidades de fala, como fonemas, palavras isoladas ou frases  \cite{fundRecFala}. 

\subsection{Reconhecedores baseados na análise acústico-fonética}

Os sistemas baseados na análise acústico-fonética decodificam o sinal de fala  baseados nas características acústicas deste sinal e na relação entre elas \cite{kluwer}. Os sistemas de análise desta classe devem considerar propriedades acústicas invariantes. Entre estas características estão a classificação entre sonoro e não sonoro, segmentação do sinal da fala, detecção das características que descrevem as unidades fonéticas e escolha do padrão que mais corresponde à sequência de unidades fonéticas.\\

Os reconhecedores baseados na análise acústico-fonética trabalham em duas etapas. O primeiro passo na análise acústico fonética é chamado de fase de segmentação e rotulagem \cite{fundRecFala}. Este passo envolve a segmentação do sinal da fala em regiões discretas, no tempo, onde as propriedades acústicas do sinal são representadas por um único fonema, ou estado. Em seguida uma ou mais etiqueta fonética é associada a cada região segmentada de acordo com as propriedades acústicas. O segundo passo para o reconhecimento tenta determinar uma palavra válida a partir da sequência de etiquetas fonéticas obtidas na fase anterior. As palavras são obtidas a partir de um determinado vocabulário, as palavras obtidas fazem sentido sintático e tem significado semântico.

\section{Processamento digital de sinais}
 
De acordo com \cite{sig} um sinal é qualquer função associada a um fenômeno físico, econômico ou social e que transporta  algum tipo de informação sobre ele. Pode ser definido como uma descrição quantitativa de um dado fenômeno. A voz é um exemplo de sinal.

Os sinais podem ser classificados de diferentes formas de acordo com suas características  e com o tipo de domínio e contradomínio. Segundo \cite{sig} esta classificação pode ser feita de acordo com as seguintes características:
\begin{enumerate}
\item Variável independente: o sinal é contínuo se a variável $t \in \mathbb{R}$ e discreto se $t \in \mathbb{Q}$. Os pontos $t_n, n \in \mathbb{Z}$ são chamados de instantes de amostragem. Sinal amostrado é o sinal discreto obtido por amostragem de um sinal contínuo.

\item Amplitude: os sinais podem ser classificados de acordo com a amplitude em :
\begin{itemize}
\item Analógicos: sinal contínuo cuja amplitude pode assumir uma gama contínua de valores;
\item Quantificados: sinal cuja amplitude pode assumir, apenas, uma gama finita de valores;
\item Digitais: sinal resultante da codificação de um sinal amostrado e quantificado. A codificação consiste em atribuir a cada valor obtido por amostragem e quantificação  um código.
\end{itemize} 
\item Duração: os sinais cujo dominio é limitado dizem-se de duração finita, os restantes são de duração infnita. Os sinais de duração finita tambem são chamados de janela.
\item Reprodutibilade: um sinal é dito determinístico se repetindo a mesma experiência obtém-se o mesmo resultado, caso isso não seja possível  então trata-se de um sinal aleatório.
\item Periodicidade: os sinais determinísticos classificam-se ainda em aperiódicos e periódicos. Os sinais aperiódicos não são repetitivos. Os sinais periódicos são repetitivos e possuem a relação $x(t) = x(t \bar{+} T) \quad \forall \quad t$, onde $T$ é o período. Quando $T < 2 \pi$ a involvente final do sinal periódico $x(t)$ não coincide com a extensão periódica do sinal base $x_b(t)$ ocorre o fenômeno chamado \textit{aliasing}. O fenômeno de aliasing é importante na conversão discreto-contínua e verifica-se no domínio da frequência.
\item Morfologia: formas simétricas a um eixo ou outro. Os sinais pares são simétricos ao eixo das ordenadas. Os sinais impares são simétricos ao eixo das abscissas.
\item Caráter : outras medidas são consideradas. Um sinal pode ter carater escalar, vetorial. Por exemplo, o sinal de saída de sensores é um sinal sensorial.
\end{enumerate}

A análise frequencial moderna é um conjunto de técnicas matemáticas e ou físicas que permite  obter o conteúdo frequêncial de qualquer sinal, a que se chama de \textit {espectro} . O processo de obtenção de espectro chama-se \textit {análise espectral}. O processo numérico usado para determinar o espectro é chamado de  \textit {estimação espectral}. A estimação espectral é feita em sinais de fonte fisíca, como a voz, durante um intervalo de tempo finito. Na prática o conteúdo frequêncial de um dado sinal não é uniforme. Assume valores significativos em intervalos chamados bandas.  A Tabela \ref{tab:banda}
mostra alguns sinais e suas bandas.

\begin{table}[H]
\centering
\caption{Bandas ocupadas por alguns sinais}
\label{tab:banda}
\smallskip
\begin{tabular}{|l|l|l|}
\hline
Sinal  & de & a\\[0.5ex]
\hline
&&\\[-2ex]
Eletrocardiograma& 0 Hz & 150 Hz \\[0.5ex]
\hline
&&\\[-2ex]
Eletroencefalograma& 0 Hz & 100 Hz\\[0.5ex]
\hline
&&\\[-2ex]
Voz & 100 Hz & 4000 Hz\\[0.5ex]
\hline
&&\\[-2ex]
Ruído do vento& 100 Hz &1000 Hz \\[0.5ex]
\hline
&&\\[-2ex]
Ruído de tremor de terra& 0.01 Hz& 10 Hz \\[0.5ex]
\hline
&&\\[-2ex]
Radiodifusão& 0.03 MHz& 3 MHz\\[0.5ex]
\hline
&&\\[-2ex]
Onda curta& 3 GHz & 30 GHz\\[0.5ex]
\hline
&&\\[-2ex]
Radar, satélite, comun. espaciais& 300 GHz & 300 THz \\[0.5ex]
\hline
&&\\[-2ex]
Luz visível& 370 THz& 770 THz \\[0.5ex]
\hline
\end{tabular}
\end{table}


 A designação de filtro habitualmente usada em referência aos sistemas lineares, deriva da possibilidade de certos sistemas eliminarem ou atenuarem fortemente certas bandas.



























%alguns exemplos de citação:\\
%\cite{berquo1980fatores}\\
%\cite{santos1980dinamica}\\
%\cite{NBR6023:2002}\\
%\cite{NBR14724:2005}\\
%\cite{NBR10520:2002}\\
%\cite{lessa2004manual}\\
%\cite{rey2000planejar}\\
%\cite{rajagopalan2003identidade}\\
%\cite{flemming1999calculo}\\
%\cite{gonccalves2}\\
%\cite{salgado2002nutriccao}
\chapter{CAPTURA DE ÁUDIO}
\label{chap:cap_audio}
\thispagestyle{plain}
\quad A captura do sinal de áudio é uma parte fundamental para o desenvolvimento de um sistema reconhecedor de fala. Existem bases de dados disponíveis para testes em que a captura
do sinal de áudio não é necessária, um vez que estas bases disponibilizam os arquivos de áudio. Um exemplo de base de dados de voz é a Aurora-1, esta base é construída por sinais de fala   limpos e degradados através de oito tipos de ruídos \cite{aurora}. Neste trabalho optamos por realizar a captura do áudio pois este também faz parte do objetivo.\\ 
O som se propaga no ambiente por meio de ondas de forma contínua no tempo e no espaço a uma velocidade média de \textit{340 metros/segundo} fazendo o ar vibrar. Esta onda sonora  é capturada por meio de um microfone como uma onda analógia e  é convertida para um sinal digital. A onda capturada é normalizada através de um filtro de passa-baixas. Circuitos que realizam esta conversão de onda são chamados de  ADC (\textit{ analog digital converter}). O tamanho das amostras, expressa em bits, é um dos fatores que determina a precisão com que o som é representado em forma digital. Outro fator importante que afeta a qualidade de som é a taxa de amostragem. O teorema de Nyquist  afirma que a frequência mais elevada que pode ser representada com precisão é, no máximo, metade da taxa de amostragem \cite{nyqui}.

\section{Bibliotecas para Captura de Áudio}
\quad Para o processo de reconhecimento de fala de qualquer tipo, primeiro é necessário capturar o sinal de áudio. A fase de captura de áudio é essencial para o bom desempenho do projeto. Existem diversas bibliotecas open-source que oferecem funções que realizam a captura e gravação de áudio, entre elas a Allegro e OpenGL, entretanto a aplicação dessas bibliotecas implica em um maior custo computacional, uma vez que estas trazem milhares de linhas de código junto com outras funções além das necessárias para a implementação deste projeto. Com base nisso, buscou-se uma alternativa que integrasse eficiência e baixo custo computacional para aplicações em áudio. 
 
\subsection{ALSA}
\quad ALSA (\textit{advanced linux sound architeture }) consiste de um conjunto de drivers do kernel, uma biblioteca, uma API e programas utilitários para o suporte de som no linux. Jaroslav Kysela iniciou o projeto ALSA porque os drives de som do kernel Linux não estavam sendo devidamente mantidos e atualizados. Após  a iniciativa mais desenvolvedores aderiram ao projeto e a estrutura da API foi refinada. ALSA foi incorporada ao kernel oficial do Linux 2.5.
A biblioteca fornecida pelo ALSA, libasound, fornece uma nomeação lógica dos dispositivos de hardware. Os nomes podem ser de dispositivos de hardware reais ou plugins \cite{linux}. Os dispositivos de hardware usam o formato $HW:i,j$, onde $i$ é o número do cartão e $j$ do dispositivo do cartão. Uma placa de som tem um buffer de hardware que armazena amostras gravadas. Quando este buffer enche, ele gera uma interrupção. O driver de som do kernel, em seguida, utiliza o acesso direto à memória  para transferir as amostras para um buffer de aplicativo na memória. O tamanho deste buffer pode ser  programado por chamadas da biblioteca ALSA. Caso o buffer seja muito grande a tranferencia geraria uma latência excessiva. ALSA resolve isso dividindo o buffer em fragmentos e transfere os dados fragmentados. A Figura \ref{fig:pcm} ilustra a repartição do buffer em periodos, quadros e amostras, onde: 
\begin{itemize}
\item Periodos: contém fragmentos de dados em um ponto no tempo.
\item Fragmentos: Menor unidade de um periodo.
\item Amostra: valores(nesse caso,  contém 2 bytes, 1 é o Menor Bit Significativo e o outro Maior Bit Significativo).
\end{itemize}
\begin{figure}[H]
\centering % para centralizarmos a figura
\includegraphics[width=10cm]{img/pcm.jpg} % leia abaixo
\caption{Buffer de aplicação. \textit{fonte:\cite{linux}}}
\label{fig:pcm}
\end{figure}

\quad A API ALSA oferece seis principais interfaces. São elas  a interface de controle, interface MIDI raw, interface de tempo, interface de sequência, interface mixer e interface de PCM. Esta última gerencia a captura e reprodução de áudio digital. 


\section{Arquivos WAVE}

\quad O formato de áudio adotado foi o WAVE. Neste tipo de formato o som é armazenado em sequências numéricas. O áudio é convertido em dados e armazenado bit a bit. O WAVE (.wav) foi criado pela IBM e pela Microsoft, nos anos oitenta e tem suporte a  uma série de resoluções de bit, taxas de amostragens e canais de áudio.  A taxa de amostragem em arquivo .wav refere-se ao número de amostras por segundo. O CD possui uma taxa de amsotragem de $44,100$, o que significa que cada segundo de áudio tem $44,100$ amostras. A quantidade de bits usada determina  quanta informação pode ser armazenada  no arquivo. A quantidade de bits também interfere na amplitude do sinal. Em uma gravação de 8 bits estará disponível 256 níveis de amplitude, variando de $0$ à $255$. Em uma gravação de 16 bits a quantidade de níveis de amplitude disponíveis passa a $65,536$, variando entre $-32,768$  até $32767$. A quantidade de 16 bits é suficiente para este projeto. 

\subsection{Cabeçalho WAVE}

\quad O cabeçalho de um arquivo .wav possui 44 bytes e é organizado como mostrado na Tabela \ref{tab:app}.

\begin{table}[H]
\centering
\caption{Formato de um cabeçalho de arquivo wave}
\label{tab:app}
\smallskip
\begin{tabular}{|l|l|l|}
\hline
Posição & Valor & Descrição\\[0.5ex]
\hline
&&\\[-2ex]
1 - 4& RIFF & Define como um arquivo RIFF \\[0.5ex]
\hline
&&\\[-2ex]
5 - 8& Tamanho do arquivo (int) & Tamanho máximo do arquivos em bytes \\[0.5ex]
\hline
&&\\[-2ex]
9 - 12 & "WAVE" & Arquivo tipo cabeçalho wave\\[0.5ex]
\hline
&&\\[-2ex]
13 - 16& "fmt" & Marca formato chunk \\[0.5ex]
\hline
&&\\[-2ex]
17 - 20& 16& Tamanho do formato dos dados \\[0.5ex]
\hline
&&\\[-2ex]
21 - 22& 1& Formato tipo PCM\\[0.5ex]
\hline
&&\\[-2ex]
23 - 24& 2 & Quantidade de canais\\[0.5ex]
\hline
&&\\[-2ex]
25 - 28& 44100 & Taxa de amostragem \\[0.5ex]
\hline
&&\\[-2ex]
29 - 32& 176400& (taxa de amostragem * bits por amostra * canais) / 8 \\[0.5ex]
\hline
&&\\[-2ex]
33 - 34&  4 & limites \\[0.5ex]
\hline
&&\\[-2ex]
35 - 36& 16 & Quantidade de bits por amostra \\[0.5ex]
\hline
&&\\[-2ex]
37 - 40& data & Marca o início da seção de dados \\[0.5ex]
\hline
&&\\[-2ex]
41 - 44& Tamanho do arquivo (dados) & Tamanho da seção de dados \\[0.5ex]
\hline
\end{tabular}
\end{table}




\chapter{PRÉ-PROCESSAMENTO}
 \thispagestyle{plain}
\label{chap:pre_proc}

\quad O processo para reconhecimento de fala pode ser divido em várias etapas. O sinal de áudio é recebido do meio externo através de um transdutor e convertido para um sinal digital
a partir deste momento devemos tratar este sinal. A  Figura \ref{fig:diaMFCC} ilustra as etapas do processo de extração de características MFCC.

\begin{figure}[H]
\centering % para centralizarmos a figura
\includegraphics[width=10cm]{img/diaMFCC.jpg} % leia abaixo
\caption{Etapas para extração de coeficientes MFCC. \textit{fonte: Autoria própria}}
\label{fig:diaMFCC}
\end{figure}

O sinal recebido deve passar pelo pré-processamento para reduzir as interferências externas do sinal e ressaltar as informações úteis. Durante a etapa de pré-ênfase o sinal é normalizado. A normalização da amplitude do sinal garante que sons em diferentes alturas sejam processados igualmente. Os períodos de silêncio do sinal são retirados para que apenas dados importantes sejam armazenados.\\ Após a etapa de pré-ênfase é realizado o janelamento do sinal, ou seja, o sinal é dividido em frames. É aplicada uma janela de Hamming para atenuar as descontinuidades causadas no início e final de cada frame.  A próxima etapa é a aplicação da Transformada Rápida de Fourier (FFT - do inglês \textit{fast fourier transform}) no sinal. A equação \ref{eqima} para obter a potência espectral.
\begin{equation}
\label{eqima}
S[k] = |X[k]|^2 = (real(X[k]))^2 +  (imaginaria(X[k]))^2
\end{equation}
 A FFT transforma um sinal do domínio do tempo para um do domínio da frequência.  A  Transformada Discreta de Fourier (DFT - do inglês \textit{discret fourier transform}) possui complexidade $O(n^2)$ e a FFT possui complexidade $O(n log n)$, por este motivo a FFT é usada em aplicações computacionais. A próxima etapa é a aplicação do banco de filtros triângulares, estes exigem uma explicação mais detalhada de como foram feitos. Esta explicação é feita em detalhes na seção \ref{sec:filt_tri}.

\section{Filtros Triângulares}
\label{sec:filt:tri}
\quad Para entendermos os filtros triângulares precisamos falar sobre a escala Mel.

\subsection{Escala Mel}
\quad Em 1937 Stanley Smithy Stevens, John Volkman e Edwin Newmann propuseram o uso de uma variável psicoacústica chamada  \textbf {pitch}  para a criação de uma escala musical perceptual de tons em intervalos igualmente espaçados, chamada escala  \textbf {mel}. A frequência ouvida pelo sistema auditivo humano é subjetiva e varia de acordo com cada indivíduo. Esta impressão subjetiva de frequência é a sensação subjetiva da intensidade ou a amplitude de um som. A escala \textit{mel} é uma escala de pitches julgados pelos ouvintes como sendo igual em distância um do outro. O ponto de referência entre esta escala e a medição de freqüência normal é definida  igualando um tom de 1000 Hz , 40 dB acima do limiar do ouvinte , com um pitch de 1000 \textit{mels}. Abaixo de cerca de 500 Hz as escalas de \textit{mel} e Hertz coincidem, acima disso intervalos cada vez maiores são julgados por ouvintes para produzir iteração igual aos pitches. A escala \textit{mel} é baseada em um mapeamento entre a frequência real e o pitch aparentemente percebido do sistema auditivo humano. Para converter uma frequência em escala \textit{mel} aplica-se a equação \ref{melscale}, onde $f$ é frequência.
\begin{equation}
\label{melscale}
M(f) = 1125 ln(1 + \frac{f}{700})
\end{equation}


A percepção humana de algumas frequências de sons complexos não podem ser individualmente dentro de certas bandas, quando uma dessas componentes cai fora da banda, chamada de banda crítica, ela pode ser identificada. Isto ocorre porque a percepção de uma frequência particular pelo sistema auditivo, por exemplo $f_0$, é influenciada pela energia da banda crítica das frequências em torno de $f_0$. O valor dessa banda varia nominalmente de $10 \%$ a $20 \%$ da frequência central do som, começando em torno de $100 Hz$ para frequências abaixo de $1 kHz$ e aumentando em escala logarítmica acima disso. Com base nestes fenômenos utiliza-se o logarítmo da energia total das bandas críticas em torno das frequências mel. A aproximação utilizada para este cálculo é a utilização de um banco de filtros espaçados uniformemente na escala mel, o banco de filtros triangulares. Os filtros \textit{mel} são definidos de acordo com a função \ref{filtro}.
\begin{equation}
\label{filtro}
%\begin{displaymath}
H_m[k] = \left\{\begin{array}{ll}
0 & k < k[m-1]\\
\displaystyle \frac{2(k-k[m-1])}{(k[m+1]-k[m-1])(k[m]-k[m-1])}, & k[m-1] \leq k \leq k[m] \\
\displaystyle \frac{2(k[m+1]-k)}{(k[m+1]-k[m-1])(k[m+1]-k[m])}, & k[m] \leq k \leq k[m+1] \\
0 & k > k[m+1]\end{array} \right.
%\end{displaymath}
\end{equation}

A fórmula acima, é utilizada para a transformar a frequência acústica de Hz em Mel, Sendo:
M = números de filtros\\
m = número do filtro, 1 <= m <= M\\
f(m) = frequência central de cada filtro\\

A Figura \ref{fig:filtro} mostra o banco de filtros usados na técnica MFCC. Cada filtro calcula a média do espectro em torno de um espectro central. Quanto maior a frequência, maior é a largura da banda.

\begin{figure}[H]
\centering % para centralizarmos a figura
\includegraphics[width=10cm]{img/filtrotriangular.jpg} % leia abaixo
\caption{Banco de filtros triângulares MFCC. \textit{fonte: \cite{pucpncc}}}
\label{fig:filtro}
\end{figure}

Para determinar matematicamente os segmentos, parte-se da frequência extremas $f_l$ e $f_h$ que são as frequências de corte do banco de filtros em Hz. Esses valores são usados para dividir o intervalo em $B+1$ partes iguais. Para obter os valores em Hz, basta aplicar a função inversa \ref{inversa}.

\begin{equation}
\label{inversa}
k[m] = \big( \frac{N}{F_s}\big) Mel^{-1} \big(Mel(f_l) + m \frac{Mel(f_h)- Mel(f_l)}{M+1}\big)
\end{equation}

onde $F_s$ é a frequência de amostragem em Hz, M é o número de filtros e N o número de amostras da FFT. $k[m]$ são as frequências digitais e $Mel^{-1}$ determina a largura do banco de filtros e é dado por
\begin{equation}
Mel^{-1}(m) = 700(e^{\frac{m}{1125}} - 1)
\end{equation}



Em seguida, obtém-se a log-energia da saída de cada um dos filtros \textit{mel}. Por fim os coeficientes MFCC são obtidos aplicando a  Transformada Discreta de Cosseno (DCT - do inglês \textit{discret cosine transform}) ao logarítmo dos coeficientes de energia obtidos no passo anterior.






























\chapter{MODELOS OCULTOS DE MARKOV}
\thispagestyle{plain}
\label{chap:hmm}
Neste capítulo é realizada a descrição do algoritmo HMM. A seção \ref{sec:topo} traz uma análise das diferentes topologias do HMM e a seção \ref{sec:3prob} mostra os três problemas a serem solucionados no algoritmo.

\quad Um modelo de Markov pode ser definido como um conjunto finito de estados ligados entre si por transições, formando uma máquina de estados. Estas transições estão ligadas por um processo estocástico .  Há ainda um outro processo estocástico associado a um modelo de Markov, que envolve as observações de saída de cada estado. Se somente as observações de saída forem visíveis a um observador externo ao processo, diz-se então que os estados estão ocultos. %, ou seja, o processo estocástico que envolve as transições de estado não é observavel.( aparentemente por isso que é oculto)

%De forma simplificada, podemos dizer que processos estocásticos são processos aleatórios que dependem do tempo. ; Um processo estocástico é uma família de variáveis aleatórias indexadas por elementos t pertencentes a determinado intervalo temporal.Wipedia acho.).

Um HMM é caracterizado por:
\begin{itemize}

\item  Um conjunto de estados $ S =  \{S_1, S_2, \ldots, S_{n-1}, S_n\} $, onde $n$ é o número de estados;

\item Função de probabilidade de estado inicial $\pi = \{\pi_i\}$ .

\begin{equation}
\pi_i = P[q_1 = S_i ]~~\textrm{ }~ 1 \leq i \leq n 
\end{equation}
onde $q_1$ é o estado inicial $(t = 1)$.

\item Função de probabilidade de transição A;

\item Função de probabilidade de símbolos de saída B.

\end{itemize}

Considerando exclusivamente processos em que as probabilidades de transição não dependem do tempo e os HMMs são de primeira ordem, um HMM é considerado de primeira ordem quando a trasição do estado depende apenas da probabilidade do estado anterior mais recente. O conjunto de probabilidades de transição $A$ é definido por: 
 \begin{equation}
A = \{ a_{ij}\} 
\end{equation}

 \begin{equation}
 a_{ij} =  P [q_{t-1} = S_i] [q_t = S_j]~~\textrm{ }~ 1 \leq i, j\leq n
\end{equation}

onde $a_{ij}$ é a probabilidade de ocorrer uma transição do estado $S_i$ para o estado $S_j$.\\
Os coeficientes $a_{ij}$ devem obedecer às seguintes regras:

\begin{equation}
a_{ij} \geq 0~~\textrm{ }~ 1 \leq i,j \leq n 
\end{equation}

\begin{equation}
\displaystyle \sum_{j=1}^n a_{ij} = 1~\textrm{ }~ 1 \leq i \leq n 
\end{equation}

A probabilidade de estar no estado $S_j$ no instante de tempo $t$ depende somente do instante de tempo $t_j$.\\

\section{HMM e a função densidade de probabilidade}
\quad Um HMM também pode ser classificado de acordo com a função densidade de probabilidade. 

\subsection{Função densidade de probabilidade}
\quad Uma variável aleatória é uma função cujo valor é um número real determinado por cada elemento em um espaço amostral. Dada uma variável aleatória $X$, dizemos que $f(x)$ é uma função densidade de probabilidade de $X$, se e somente se $f(x)$ atender as seguintes condições:

$$
\displaystyle f(x) \geq 0  \qquad a < x < b
$$

\begin{equation}
 \int_a^b f(x)dx = 1 
\end{equation}

\subsection{HMM Discreto}
\quad O número de possíveis símbolos de saída é finito \cite{fundRecFala}.
 A probabilidade de emitir o símbolo $V_k$ no estado $S_i$ é dada por $b_i(k)$. As propriedades da função de probabilidade $B$ são:

$$
 \displaystyle  b_i (k) \geq 0 \qquad 1 \leq i \leq n  \qquad 1 \leq k \leq K
$$

\begin{equation}
\displaystyle \sum_{k=1}^K b_i (k) = 1 \qquad 1 \leq i \leq n 
\end{equation}

As observações são discretas por natureza ou discretizadas através de uma técnica de quantização vetorial, gerando assim \textit {codebooks}.
 
\subsection{HMM Contínuo}
\quad A função densidade de probabilidade é contínua. Geralmente uma função densidade elipticamente simétrica, tal como a função densidade de probabilidade Gaussiana \cite{fundRecFala}.
 As observações são contínuas e a FDP contínua é  usualmente modelada como uma mistura finita de matrizes gaussianas multidimensionais.
% DEFINIR AQUI A FDP A SER USADA (PROVAVELMENTE A GAUSSIANA CITADA EM  \cite{fundRecFala})
\subsection{HMM Semicontínuo}
\quad O modelo é um caso intermediário entre contínuo e o discreto. O conjunto função densidade probabilidade é o mesmo usado para todos os estados e todos os modelos. A probabilidade de emissão dos símbolos de saída é dada por :


\begin{equation}
\displaystyle b_j(O_t) =  \sum_{V_k \in \eta (O_t)}^-  c_j (k) f (O_t | V_k)   \qquad 1 \leq j \leq n 
\end{equation}
 onde:\\
$O_t$ é o vetor de entrada\\
$\eta(O_t)$ é o conjunto das funções densidade de probabilidade que apresentam os $M$ maiores valores de $f (O_t | V_k)$, $ 1 \leq M \leq K$\\
$K$ é o número de funções densidade de probabilidade, ou seja, os símbolos de saída\\
$V_k$ é o $k$-ésimo símbolo de saída\\
$ c_j (k)$ é a probabilidade de emissão do símbolo $V_k$ no estado $S_j$\\
$f (O_t | V_k)$ é o valor da $k$-ésima função densidade de probabilidade.



\section{Topologia}
\label{sec:topo}

\quad Uma maneira de classificar um HMM é de acordo com a estrutura de transição da matriz $A$ da cadeia de markov. Existem vários modelos de HMM, a Figura \ref{fig:topohmm} ilustra os  principais modelos de acordo com \cite{fundRecFala}. O ergódico totalmente conectado onde qualquer estado pode ser alcançado com um único passo, o modelo de caminhos paralelos e o modelo "left-right", também chamado de modelo Bakis.%\\ *********COLOCAR AQUI UM MODELO DE BAKIS FAZER A MATRIZ
\begin{figure}[H]
\centering % para centralizarmos a figura
\includegraphics[width=10cm]{img/topohmm.jpg} % leia abaixo
\caption{Exemplo de topologias de HMM. a) Modelo ergódico b) Modelo esquerda-direita c) Modelo esquerda-direita paralelo. \textit{\cite{fundRecFala}}}
\label{fig:topohmm}
\end{figure}

\section{Os problemas a serem resolvidos}
\label{sec:3prob}
\quad O HMM possui três problemas básicos, que são:
\begin{enumerate}
\item \textbf{Problema de avaliação:} Dada a sequência de observação $O = (o_1, o_2, o_3, ..., o_n)$ e o modelo $\lambda = (A, B, \pi)$, como calcular eficientemente $P(o| \lambda)$.
\item \textbf{Problema da busca da melhor sequência de estados.}
\item \textbf{Problema de treinamento:} como ajustar os parâmetros do modelo $\lambda(A, B, \pi)$ para maximizar $P(o|\lambda)$.
\end{enumerate}

O problema 1, ou seja, o problema da avaliação pode ser solucionado através do procedimento \textit{Forward-Backward}. O segundo problema é solucionado com a aplicação do algoritmo de \textit{Viterbi} e o terceiro e último problema pode ser otimizado aplicando um procedimento iterativo como o método de \textit{Baum-Welch}. Nas seções \ref{secFB}, \ref{secViterbi} e \ref{secBW} faz-se uma explicação sobre os procedimentos para a solução dos problemas 1, 2 e 3 respectivamente.

\subsection{Foward-Backward}
\label{secFB}
Com a resolução do problema 1 podemos responder à algumas perguntas, se dado um modelo e uma sequência de observações, como podemos  saber de que a sequência observada foi produzido pelo modelo? ou, podemos ver essa  solução de outra forma,  um modelo é satisfatório para determinada entrada de observações?
 \begin{itemize}
\item Inicialização:\\
\begin{equation}
\displaystyle  \alpha_1 (i) = \pi_i b_i (O_1), \qquad 1 \leq i \leq N \\
\end{equation}

\item Indução:\\
$
\displaystyle \alpha_t + 1 (j) = \sum\limits_{ i = 1}^{N} \Big[\alpha_t(i) a_{ij} \Big]b_j (O_t + 1), \qquad 2 \leq t \leq T\\
%\displaystyle \Psi_t (j) = arg\max\limits_{1 \leq i \leq N}\Big[\delta_{t-1} (i)a_{ij}\Big], \qquad 1\leq j \leq N
$

\item Término:\\
$
P(O| \lambda) =  \sum\limits_{ i = 1}^{N} \Big[\alpha_t(i) \Big]\\
$

\end{itemize}

\subsection{Viterbi}
\label{secViterbi}
O algoritmo de Viterbi é um algoritmo de programação dinâmica usado para encontrar a sequência de estados ocultos ótima. Dado uma sequência de estados ocultos de um HMM, o algoritmo de viterbi calcula a melhor sequência de estados baseados nas probabilidades de transição. Este algoritmo foi proposto em 1967 por Andrew Viterbi para a decodificação de códigos convolucionais em links de comunicação ruidosos.  O algoritmo também possui aplicações  em redes CDMA e GSM, modem dial-up, satélites, síntese de fala, linguística computacional e bioinformática. Em telecomunicação, um código convolucional é um tipo de código corretor de erro em que cada conjunto de $m$ símbolos é transformado em um conjunto de $n$ símbolos.\\
Algoritmo
\begin{itemize}
\item Inicialização:\\
\begin{equation}
\delta_1 (i) = \pi_i b_i (O_1), \qquad 1 \leq i \leq N \\
\end{equation}
$
\quad \Psi_1 (i) = 0
$

\item Recursão:\\
\begin{equation}
\displaystyle \delta_t (j) = \max\limits_{1 \leq i \leq N} \Big[\delta_{t-1}(i) a_{ij} \Big]b_j (O_t), \qquad 2 \leq t \leq T\\
\end{equation}
\begin{equation}
\displaystyle \Psi_t (j) = arg\max\limits_{1 \leq i \leq N}\Big[\delta_{t-1} (i)a_{ij}\Big], \qquad 1\leq j \leq N
\end{equation}

\item Término:\\
\begin{equation}
\displaystyle P^* =\max\limits_{1 \leq i \leq N}  \Big[\delta_{T(i)}\Big] 
\end{equation}
\begin{equation}
\displaystyle G^*_T = arg\max\limits_{1 \leq i \leq N}  \Big[\delta_{T(i)}\Big]
\end{equation}


\end{itemize}





\subsection{Baum-Welch}
\label{secBW}
Não existe uma maneira conhecida de resolver analiticamente o conjunto de parâmetros para um dado modelo de forma que seja maximizada a probabilidade da seqüência de observações. No entanto um procediemento iterativo como o método de Baum-Welch  permite escolher $\lambda = (A, B, \pi)$ tal que $P(O|\lambda)$ é maximizada localmente. O algoritmo Baum-Welch é apresentado em termos das variáveis $\alpha_t$ e $\beta_t$ dos algoritmos \textit {forward} e \textit{backward} respectivamente, e segundo  \cite{artRabiner} é o mais indicado para a estimação dos parâmetros do HMM. A re-estimação dos parâmetros $a_{ij}$ e $b_{ij}$ para uma dada sequência de observações através do método de Baum-Welch é descrita da seguinte forma por [19].\\
Para uma única sequência de observações $O = {O_1, O_2, \ldots, O_T}$ a re-estimação da probabilidade de transição do estado $i$ para o estado $j$ da matriz de transição de estados $A$ é dada por:


\begin{equation}
\displaystyle a_{ij} = \frac{\sum_{t=1}^{T-1} a_t(i)a_{ij}b_j (O_{t+1}) \beta_{t+1}(j)}{\sum_{t=1}^{T-1} a_t(i)\beta_t(j)}
\end{equation}


Para os HMM’s discretos, a quantidade de símbolos de saída é finita. Também para uma
única elocução, a re-estimação da função de probabilidade para que um estado $q_i$ emita um símbolo $O_t = V_k $ é obtida por:

\begin{equation}
\displaystyle  b_i(k) = \frac{\sum_{t=1}^{T-1} a_t(i)\beta_t(j) t.q. \quad O_t = V_k}{\sum_{t=1}^{T-1} a_t(i)\beta_t(j)}
\end{equation}

onde $ \displaystyle \qquad b_i(k) \geq 0, \qquad 1\leq i \leq N, \qquad 1 \leq k \leq M, \qquad \sum_{k=1}^M b_i(k) = 1, \qquad 1 \leq i \leq N$















\chapter{REDES NEURAIS ARTIFICIAIS}
\label{chap:ann}
%quad = paragrafo \\ pula linha
\quad Redes neurais artificiais são estruturas matemáticas capazes de aprender, memorizar e generalizar determinadas situações e problemas a ela apresentado. Rede neural é inspirada no cérebro e é formada por neurônios que se ligam entre si e de acordo com uma determinada função, realizam sinapses entre si \cite{hay}.\\

Em 1943 o psiquiatra e neuroanatomista Warren McCulloch e o matemático Walter Pitts propuseram um modelo matemático de um neurônio artificial. O modelo era uma simplificação do neurônio biológico representado na Figura \ref{fig:neur}. 

\begin{figure}[H]
\centering % para centralizarmos a figura
\includegraphics[width=10cm]{img/neuronio.jpg} % leia abaixo
\caption{Estrutura básica de um neurônio biológico. \textit{fonte:\cite{barroso}}}
\label{fig:neur}
\end{figure}

Para representar os dendritos, o modelo usou $n$ terminais de entrada de informações $x_1, x_2, x_3, \dots, x_{n-1}, x_n$ e um terminal de saída $y$ representando o axônio. As sinapses são  simuladas de acordo com um coeficiente ponderador, a sinapse só ocorre quando a soma ponderada dos sinais de entrada ultrapassa um limiar pré-definido. Este limiar é chamado de função de ativação e foi definido de forma Booleana. A Figura \ref{fig:ann} mostra o neurônio artificial.

\begin{figure}[H]
\centering % para centralizarmos a figura
\includegraphics[width=10cm]{img/neura.jpg} % leia abaixo
\caption{Neurônio artificial proposto por McCulloch e Pitts. \textit{\cite{fia}} }
\label{fig:ann}
\end{figure}

A equação \ref{eq:neu} é a equação da saída $y$ do neurônio.
\begin{equation}\label{eq:neu}
y = f \Big(\sum_{i=1}^{n}x_iw_i +b\Big)
\end{equation}
Onde $n$ é o número de entradas do neurônio, $w_i$
 é o peso associado à entrada $x_i$ e $f$ é a função de ativação utilizada. 


\section{Classificação de Redes Neurais Artificiais}
\label{sec:class}
\quad Redes neurais podem ser classificadas de acordo com a topologia ou a forma de aprendizagem. Estas classificações são explicadas nas seções \ref{sec:top} e \ref{sec:lear} respectivamente.

\subsection{Topologia}
\label{sec:top}
As RNAs podem ser classificadas de acordo com sua topologia em:
\begin {enumerate}
\item Perceptron de Camada Única: É utilizado para classificação linear, ou seja, utilizada em problemas que sejam linearmente separáveis. A Figura \ref{fig:ann1} mostra um exemplo de perceptron de camada única;
\begin{figure}[H]
\centering % para centralizarmos a figura
\includegraphics[width=10cm]{img/rna1.jpg} % leia abaixo
\caption{Perceptron de camada única.}
\label{fig:ann1}
\end{figure}
\item Perceptron de Múltiplas Camadas: Possui várias camadas, e possuem, camadas ocultas, onde dentro delas, esta inserindo neurônios ocultos, ou seja recursos computacionais.  A Figura \ref{fig:annm} mostra um exemplo de perceptron de várias camadas;

\begin{figure}[H]
\centering % para centralizarmos a figura
\includegraphics[width=10cm]{img/annm.png} % leia abaixo
\caption{Perceptron de múltiplas camadas.}
\label{fig:annm}
\end{figure}

\item Redes Recorrentes: Possui pelo menos um laço de realimentação, pode ser implementada tanto no perceptron de camada única,  como, no perceptron de multicamadas.  A Figura \ref{fig:annrec} mostra um exemplo de perceptron recorrente;

\begin{figure}[H]
\centering % para centralizarmos a figura
\includegraphics[width=10cm]{img/rnarea.jpg} % leia abaixo
\caption{Perceptron recorrente.}
\label{fig:annrec}
\end{figure}


\end{enumerate}

\subsection{Tipos de Aprendizagem}
\label{sec:lear}
Outra maneira de classificar redes neurais é de acordo com o tipo de aprendizagem. Esta pode ser:
\begin {enumerate}
\item Aprendizagem Supervisionada: Há uma amostra  que será comparada ao ambiente.
\item Aprendizagem Não Supervisionada: Procura-se um padrão, sem a ajuda de uma amostra de comparação. Essas redes são chamadas de mapas auto-organizáveis, estes são explicados na seção \ref{sec:som}.
\end {enumerate}


\section {SOM - Self Organizing Maps}
\label{sec:som}
\quad  De acordo com \cite{ia}  um mapa auto-orgánizavel é estruturado pelo neurônio vencedor de uma competição, ditada por uma função discriminante de cada iteração, também chamada de época. Essa competição pode ser em neurônio contra todos os neurônios da rede, ou, neurônio contra um grupo de neurônios da rede. Um das metas de um mapa auto-organizável é classificar os dados de entrada competindo entre si. O algoritmo possui um conjunto de regras de natureza local. O termo local significa que a modificação aplicada ao peso sináptico de um neurônio é confinada à vizinhança imediata daquele neurônio.

\section { Alguns principios intuitivos de auto-organização}
\quad A organização da rede acontece em dois níveis diferentes, que interagem entre si na forma de um laço de realimentação.  De acordo com \cite{hay} os dois niveis são:
\begin{itemize}
\item  Atividade: Certos padrões de atividade são produzidos por uma determinada rede em resposta a sinais de entrada.
\item Conectividade: Forças de conexão dos pesos sinápticos da rede  são modificadas em resposta a sinais neurais dos padrões de atividade. 
\end{itemize}

Pode-se citar, ainda de acordo com \cite{hay}, que mapas auto-organizáveis possuem os seguintes princípios:
\begin{itemize}
\item  Modificações dos pesos  sinápticos tendem a se auto-amplificar. Para estabilizar o sistema, deve haver alguma forma de competição por recursos limitados. Especificamente, um aumento na força de algumas sinapses da rede deve ser compensados por uma redução em outras sinapses.
\item  A limitação de recursos leva à competição entre sinapses e com isso à seleção das sinapses  que crescem mais vigorosamente  às custas das outras sinapses.
\item As modificações em pesos sinápticos tendem a cooperar.
\item Ordem e estrutura nos padrões de informação representam informação redundante que é adquirida pela rede neural na forma de conhecimento, que é um pré-requisito necessário para a aprendizagem.
\end{itemize}

\section{Mapa Auto-Organizável de Kohonen}
\label{sec:kohonen}
Os mapas auto-organizáveis foram desenvolvidos por Teuvo Kohonen em 1981 e fazem parte de um grupo de redes neurais baseadas em modelos de competição \cite{fia}. Uma característica importante destes mapas é que eles utilizam treinamento não supervisionado, os neurônios competem entre si e ajustam  seus pesos com base nesta competição. O principal objetivo dos mapas auto-organizáveis de Kohonen é agrupar os dados de entrada que são semelhantes entre si formando classes ou agrupamentos denominados \textit{clusters}. Segundo \cite{agua} os mapas de Kohonen podem ser aplicados para problemas não lineares de alta dimensionalidade, como por exemplo: processamento de sinais, demodulação e transmissão de sinais, extração de características e classificação de imagens e padrões acústicos, química e medicina. \\
Baseado no aprendizado competitivo, os mapas de Kohonen são do tipo \textit{winner-takes-all}, ou seja, \textit{o vencedor leva tudo}. Os neurônios de saída competem para serem ativados e a cada iteração apenas um neurônio é ativado. O funcionamento do SOM pode ser compreendido em diferentes etapas. A etapa competitiva na qual se define o neurônio mais adequado, chamado de \textbf{BMU} (\textit{do inglês Best Matching Unit}). A escolha da melhor correspondência entre o vetor de entrada e o vetor peso é feita por critério da menor distância euclidiana entre o vetor de pesos por ela armazenado e o vetor de entrada de acordo com a equação \ref{eq:euc} 
\begin{equation}
\label{eq:euc}
i(x)= argmin \big|x - w_j \big| \quad j = 1,2, \ldots n
\end{equation}
Onde $i(x)$ é a representação do neurônio da entrada $x$, e $w_j$ é o vetor peso. A função da  distância Euclidiana usada para 
quantificar a semelhança entre os vetores da rede é definida pela equação \ref{eq:euc2}.
\begin{equation}
\label{eq:euc2}
D_E = \sqrt{(x_1 - y_1)^2 + (x_2 - y_2)^2 + \ldots + (x_n - y_n)^2}
\end{equation}
Onde $x_n$ são as coordenadas dos vetores de entrada e $y_n$ são as coordenadas dos vetores  pesos  das redes auto-organozáveis.\\
Na etapa cooperativa os vizinhos são definidos dentro de uma distância obtida a partir da BMU. O processo de treinamento consiste na otimização da distância entre os neurônios. A vizinhança topológica é definida por meio da interatividade entre os neurônios. Um neurônio ativado tende a excitar os neurônios em sua vizinhança imediata. Cada iteração de atualização dos valores e distâncias da rede é chamada de época. As épocas constituem a  etapa adaptativa.



%\chapter{IMPLEMENTAÇÃO}
\quad Neste capítulo é realizada uma discussão sobre a implementação computacional dos algoritmos estudados. O ambiente de desenvolvimento para todos os algoritmos foi o mesmo. O sistema operacional foi o GNU/Linux Ubuntu 15.0.

\quad A implementação foi dividida em etapas. A primeira etapa consiste na captura do áudio, esta foi realizada com a aplicação da biblioteca \textit{alibsound.h}. Esta usa a interface PCM para a modulação do sinal. Nesta fase é necessário checar os parâmetros de hardware através de funções disponibilizadas pela ALSA, documentação disponível em http://www.alsa-project.org/alsa-doc/alsa-lib/pcm.html. Após a captura da onda foi realizado o tratamento da mesma. Foram implementadas funções para aplicação da FFT, DCT, dividir o sinal em frames, aplicar janela de Hamming e por fim extrair caraterísticas do sinal. Nesta fase as principais dificuldades encontradas foram a manipulação de fórmulas matemáticas complexas e o baixo nível de programação. Após a extração das características MFCC é realizada a comparação dos padrões.

\begin{figure}[H]
\centering % para centralizarmos a figura
\includegraphics[width=10cm]{img/diaraizatcc.png} % leia abaixo
\caption{Organização dos módulos implementados.}
\label{fig:diatcc}
\end{figure}


\begin{figure}[H]
\centering % para centralizarmos a figura
\includegraphics[width=10cm]{img/getlist.jpg} % leia abaixo
\caption{leg1}
\label{fig:getlist}
\end{figure}

\begin{figure}[H]
\centering % para centralizarmos a figura
\includegraphics[width=10cm]{img/escinaudio.jpg} % leia abaixo
\caption{leg2.}
\label{fig:inau}
\end{figure}




\section{Comparação de Padrões}
\quad Uma vez havendo extraído as caraterísticas MFCC podemos realizar a comparação entre o padrão de entrada e os
padrões armazenados. Esta comparação pode ser realizada por diferentes algoritmos.




\subsection{Método Determinístico}
\label{sec:det}
\quad O método de comparação de padrão determinístico compara a correlação entre dois vetores de características MFCC, quanto menor o resultado retornado pela função mais parecidos são os padrões. Este algoritmo também foi implementado em linguagem C. Neste método é usado um limiar de correlação para definir se os padrões são iguais. 
\begin{figure}[H]
\centering % para centralizarmos a figura
\includegraphics[width=10cm]{img/struct.jpg} % leia abaixo
\caption{leg3.}
\label{fig:str}
\end{figure}



\subsection{SOM}
\label{sec:som}
\quad A implementação do SOM  foi realizada em linguagem de programação C. A principal dificudade encontrada na implementação deste método foi a manipulação do mapa e seus neurônios. O gerenciamento da memória na alocação e liberação de vários ponteiros também se mostrou bastante complexa. O mapa auto-organizável requer  mais memória e tempo na execução do que o método determinístico citado na \ref{sec:det}.


\begin{figure}[H]
\centering % para centralizarmos a figura
\includegraphics[width=10cm]{img/structneuron.jpg} % leia abaixo
\caption{\textit{Structs} para mapa e neurônio.}
\label{fig:strneu}
\end{figure}



\subsection{HMM}
\label{sec:hmm}
\quad O algoritmo HMM foi implementado em linguagem computacional Python 2.7. Python é uma linguagem de tipagem dinâmica de alto nível, isto facilta na manipulação das estruturas de dados usadas pelo algoritmo.


\begin{figure}[H]
\centering % para centralizarmos a figura
\includegraphics[width=10cm]{img/classhmm.jpg} % leia abaixo
\caption{Classe principal do HMM.}
\label{fig:classhmm}
\end{figure}


\chapter{RESULTADOS}
\thispagestyle{plain}
\label{chap:anal}

\quad Foram estudados e avaliados três diferentes abordagens para o reconhecimento de palavras, sempre levando em consideração o contexto da aplicação e o caso de teste. 
O trabalho foi dividido em duas etapas na primeira etapa ocupou-se da captura e tratamento do sinal e extração de características.\\ 
\quad Foram avaliados um método estocástico (HMM), um método determinístico (correlação entre vetores) e um método de inteligência articial (SOM). 
\begin{table}[H]
\centering
\caption{Taxa de acertos do método determinístico}
\label{tab:comp}
\smallskip
\begin{tabular}{|l|l|l|l|}
\hline
 Palavra & Ambiente Silencioso & Ambiente Ruidoso\\[0.5ex]
\hline

Ajuda & 70\% &  60\%\\[0.5ex]
\hline

Assalto & 80\% &  40\% \\[0.5ex]
\hline

Ladrão & 60\% & 34\% \\[0.5ex]
\hline

Polícia & 90\% & 60\%\\[0.5ex]
\hline

Socorro & 90\% & 80\%\\[0.5ex]
\hline

Média & 78\% & 54,8\% \\[0.5ex]
\hline
\end{tabular}
\end{table}

A tabela \ref{tab:comp} mostra a porcentagem de acerto para cada palavra usando o método determinístico. Cada palavra foi inserida dez vezes por um número aleatório de pessoas  em um ambiente com pouco ruído a média da taxa de acertos foi de 78\% e em um ambiente com muito ruído esta taxa caiu para 66,4\%.

\begin{table}[H]
\centering
\caption{Taxa de acertos do método determinístico}
\label{tab:comp2}
\smallskip
\begin{tabular}{|l|l|l|}
\hline
 Palavra & Isolado & Contínuo \\
\hline

Ajuda & 60\% &  35\% \\
\hline

Assalto & 40\% &  35\% \\
\hline

Ladrão & 30\% & 34\%\\
\hline

Polícia & 60\% & 37\% \\
\hline
Socorro & 80\% & 40\%\\
\hline

Média & 59\% & 36,6\% \\
\hline
\end{tabular}
\end{table}


A tabela \ref{tab:comp2} mostra a porcentagem de acerto para cada palavra usando o método determinístico com abordagem contínua. Cada palavra foi inserida vinte vezes por um número aleatório de pessoas  em um ambiente ruidoso. A média da taxa de acertos foi de 45,1\% .

\begin{table}[H]
\centering
\caption{Taxa de acertos do algoritmo de Kohonen}
\label{tab:comp3}
\smallskip
\begin{tabular}{|l|l|l|}
\hline
 Palavra & Ambiente Silencioso & Ambiente Ruidoso\\
\hline

Ajuda & 70\% &  65 \% \\
\hline

Assalto & 80\% &  55 \% \\
\hline
Ladrão & 75\% &  65 \% \\
\hline

Polícia & 75 \% & 65\% \\
\hline

Socorro & 80\% & 70\% \\
\hline

Média & 76\% & 64\% \\
\hline
\end{tabular}
\end{table}

A tabela \ref{tab:comp3} mostra  a porcentagem de acerto para cada palavra usando o algoritmo SOM. Cada palavra foi inserida dez vezes por um número aleatório de pessoas  em um ambiente com pouco ruído a média da taxa de acertos foi de 76\% e em um ambiente com muito ruído esta taxa caiu para 64\%.

\begin{table}[H]
\centering
\caption{Comparação entre os métodos}
\label{tab:comp5}
\smallskip
\begin{tabular}{|l|l|l|l|}
\hline
 -- & Determinístico & HMM & SOM\\[0.5ex]
\hline
&&&\\[-2ex]
Taxa de acerto & 66,4\% &  -- &  70\%\\[0.5ex]
\hline
&&&\\[-2ex]
Tempo de execução & rápido &  viciada &  médio \\[0.5ex]
\hline
&&&\\[-2ex]
Custo computacional & baixo & alto & alto\\[0.5ex]
\hline
\end{tabular}
\end{table}

A tabela \ref{tab:comp5} traz uma comparação entre os métodos estudados. O algorimto HMM, como foi explicado no capítulo \ref{chap:hmm}, é baseado em probabilidade e nas transições de um estado a outro. Para garantir a homogeneidade do sistema
os valores de transição dos estados são iguais isso gera um vício. Sempre que uma palavra é identificada não ocorre transição de estados, retornando sempre o estado anterior.

Para o caso estudado neste trabalho podemos concluir que o  método determinístico foi o que se mostrou mais eficiente, pois este consome menos recursos computacionais e possui uma taxa de acerto satisfatória.















































 


\chapter{CONCLUSÃO}
\thispagestyle{plain}
\label{chap:conc}
\quad O presente trabalho teve como objetivo estudar algoritmos para reconhecimento de palavras isoladas em fluxo contínuo. Os métodos estudados deveriam ter a capacidade de reconhecer  uma palavra isolada independente de locutor, livre de contexto em um ambiente qualquer. Também deveria ser considerado os rescursos computacionais usados. Uma solução que requer pouca memória e com rápido tempo de processamento. Com base nisto podemos concluir que o método determinístico foi o que apresentou melhores resultados, pois o tempo de execução deste é baixo e requer poucos recursos computacionais podendo ser utilizado em
dispositivos com baixo poder de processamento.

\quad Durante a etapa de captação e processamento do sinal a qualidade do hardware usado tem grande impacto sobre as caratecterísticas extraídas. Em um sistema ideal os filtros devem ser implementados em hardware o que garante maior processamento e robustez aos ruidos. As caraterísticas usadas também devem ser analisadas, métodos como \textit{PNCC, RASTA-PLP, PLP, LPC, DBNF} são exemplos de descritores de sinal, tal como MFCC, e podem ser analisados em trabalhos futuros.

\quad A comparação entre o padrão buscado e os padrões armazenados é uma fase independente, levando em consideração as 
características usadas para o pré-processamento do sinal, ou seja, é possível aplicar diferentes algoritmos nesta fase. As técnicas 
aplicadas durante todo o processo de reconhecimento que classificam o sistema de reconhecimento. Geralmente são empregados redes neurais artificiais ou  algoritmos probabilísticos, como o HMM. Também são aplicadas técnicas determinísticas onde a  classificação é realizada através de  fórmulas matemáticas, estas possuem processamento rápido, porém são mais dependentes da variação do sinal (timbre de voz, altura da voz) e sensíveis ao ruído.\\

\quad Por fim gostariamos de sugerir para trabalhos futuros, além dos  descritores de voz citados acima, a implementação de um algoritmo para detecção de atividade de voz como citado em \cite{vad}, RNAs multicamadas e o Modelo de Misturas Gaussianas (GMM).


%capitulo intitulado conclusao e obrigatorio
%\include{cap/05_conclusao}


% ---
% Finaliza a parte no bookmark do PDF, para que se inicie o bookmark na raiz
% ---
\bookmarksetup{startatroot}% 

% ----------------------------------------------------------
% ELEMENTOS PÓS-TEXTUAIS
% ----------------------------------------------------------
\postextual

% ----------------------------------------------------------
% Referências bibliográficas
% ----------------------------------------------------------
\bibliography{bib/tfg_referencias}


% ----------------------------------------------------------
% Glossário
% ----------------------------------------------------------
%\glossary
%
% ----------------------------------------------------------
% Apêndices e Anexos
% ----------------------------------------------------------
%\begin{apendicesenv}
% include(extra/00_apendice)
%\end{apendicesenv}
%\begin{anexosenv}
%\include{extra/01_anexo}
%\end{anexosenv}


%finaliza o documento 
\end{document}


